\documentclass{article}
\usepackage{graphicx} % Required for inserting images
\usepackage{polski}
\usepackage[utf8]{inputenc}
\usepackage[T1]{fontenc}
\usepackage{hyperref}
\usepackage{amsmath}

\title{Algorytmy rekomendacyjne}
\author{Roman Kapuscik}
\date{June 2024}

\begin{document}

\maketitle

\section{Wprowadzenie}
Algorytmy rekomendacyjne stanowią nieodzowny element naszego codziennego życia i są kluczowe dla funkcjonowania wielu platform internetowych, od sklepów online po serwisy streamingowe muzyki i filmów. Umożliwiają one personalizację doświadczeń użytkowników poprzez proponowanie produktów lub treści, które mogą ich zainteresować.

\section{Zastosowanie w muzyce cyfrowej i grach komputerowych}
Algorytmy rekomendacyjne odgrywają kluczową rolę w muzyce cyfrowej i grach komputerowych, dostosowując treści do indywidualnych preferencji użytkowników i umożliwiając personalizację doświadczeń. Dzięki nim użytkownicy mogą odkrywać nowe utwory muzyczne i gry, które mogą im się spodobać na podstawie ich wcześniejszych wyborów i zachowań.
Rekomendacje te przyczyniają się do zwiększenia zaangażowania użytkowników, co z kolei skutkuje spędzaniem większej ilości czasu na platformach muzycznych i gamingowych. Ponadto algorytmy te mogą zwiększyć sprzedaż muzyki i gier, ponieważ są w stanie sugerować produkty, które najprawdopodobniej zainteresują konkretnego użytkownika. W ten sposób algorytmy rekomendacyjne optymalizują sprzedaż na platformach muzycznych i gamingowych.

\section{Popularne algorytmy rekomendacyjne}
Algorytmy rekomendacyjne są nieodzownym elementem współczesnego Internetu, pomagając odkrywać nowe treści i produkty. Jednym z najpopularniejszych typów algorytmów jest filtracja kolaboratywna, która analizuje wzorce zachowań grupy użytkowników, aby przewidywać preferencje indywidualnych użytkowników. Inny powszechnie stosowany algorytm to filtracja treściowa, skupiająca się na analizie cech produktów i dopasowywaniu ich do profilu użytkownika.
Wiele platform internetowych stosuje podejścia hybrydowe, łączące różne typy algorytmów, takie jak historia przeglądania użytkownika i analiza semantyczna treści, aby dostarczać bardziej precyzyjne rekomendacje. W ostatnich latach popularność zyskały zaawansowane techniki AI, wykorzystujące uczenie maszynowe do tworzenia coraz bardziej precyzyjnych i skutecznych rekomendacji, uwzględniających kontekst i nastrój użytkownika. Algorytmy rekomendacyjne mają szeroki zakres zastosowań, od mediów po medycynę i produkcję, a ich wpływ na nasze życie społeczne i indywidualne jest coraz bardziej widoczny.

\section{Złożoność obliczeniowa}
Złożoność obliczeniowa algorytmów rekomendacyjnych jest kluczowym zagadnieniem, określającym ilość operacji potrzebnych do wykonania programu. Obejmuje ona zarówno złożoność obliczeniową, odnoszącą się do liczby operacji, jak i złożoność pamięciową, dotyczącą wymaganej ilości pamięci operacyjnej. Aby określić złożoność obliczeniową, liczy się operacje wykonywane przez algorytm, takie jak inicjalizacja zmiennych czy wykonywanie działań, używając funkcji \( f(n) \), która zwraca liczbę operacji w zależności od ilości danych \( n \).
W praktyce stosuje się oszacowania złożoności za pomocą notacji \( O \) (wielkie O), \( \Omega \) (omega) i \( \Theta \) (theta), które umożliwiają górne, dolne lub dokładne oszacowanie złożoności. Notacja \( O \) koncentruje się na najważniejszym wyrazie funkcji, pomijając współczynniki, podczas gdy \( \Omega \) i \( \Theta \) służą odpowiednio do dolnego i dokładnego oszacowania. Algorytmy rekomendacyjne mogą mieć różne złożoności, od stałej \( O(1) \), przez liniową \( O(n) \), kwadratową \( O(n^2) \), aż po logarytmiczną \( O(\log n) \) i pierwiastkową \( O(\sqrt{n}) \), co wpływa na ich wydajność przy przetwarzaniu dużych zbiorów danych.

\section{Wnioski}
Algorytmy rekomendacyjne i związane z nimi techniki, takie jak analiza koszykowa, są kluczowe dla funkcjonowania współczesnego świata cyfrowego, przyczyniając się do personalizacji doświadczeń online. W miarę rozwoju technologii, algorytmy te stają się coraz bardziej zaawansowane, co pozwala na jeszcze lepsze dostosowanie rekomendacji do indywidualnych preferencji użytkowników. To z kolei prowadzi do większego zaangażowania użytkowników i optymalizacji sprzedaży na różnych platformach, od e-commerce po muzykę cyfrową i gry komputerowe.

\section{Referencje}
\begin{itemize}
    \item \url{https://en.wikipedia.org/wiki/Recommender_system}
    \item \url{https://en.wikipedia.org/wiki/Market_basket_analysis}
    \item \url{https://en.wikipedia.org/wiki/Collaborative_filtering}
    \item \url{https://en.wikipedia.org/wiki/Content-based_filtering}
    \item \url{https://en.wikipedia.org/wiki/Computational_complexity}
\end{itemize}
\end{document}
